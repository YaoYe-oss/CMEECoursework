\documentclass[11pt]{article}
\usepackage[a4paper, margin=1in]{geometry}
\usepackage{setspace}
\usepackage{lineno}
\usepackage{graphicx}
\usepackage{amsmath}
\usepackage{amssymb}
\usepackage{natbib}
\usepackage{hyperref}
\usepackage{booktabs}
\usepackage{longtable}
\usepackage{parskip} % Ensures no indentation and adds spacing between paragraphs
\usepackage{float} % Allows precise figure placement

% Enable line numbering
\linenumbers
\onehalfspacing

\begin{document}

% Title Page
\title{Mechanistic and Polynomial Models Exhibit Systematic Differences in Predicting Bacterial Growth Dynamics}
\author{Yao Ye \\
Imperial college London}
\date{\today}
\maketitle

\begin{center}
\textbf{Word Count: 2602}
\end{center}

\newpage
\section*{Abstract}
\thispagestyle{empty} % Removes page number from Abstract page
Understanding the dynamics of the microbial population is essential in various fields, such as food safety, biotechnology, and environmental microbiology \citep{Huang2013}. I analyze the performance of four mathematical models—two mechanical (Logistic and Gompertz) and two phenomenological (Quadratic and Cubic)—in characterizing bacterial growth under diverse cultivation settings. Models were fitted using non-linear least squares approaches on a dataset comprising bacterial growth measurements from several species. Model evaluation was performed utilizing statistical metrics, including the Akaike Information Criterion (AIC), Bayesian Information Criterion (BIC), and the coefficient of determination ($R^2$). The results demonstrate that the Gompertz model offers superior fits for bacterial growth curves characterized by significant lag phases and asymptotic plateaus, whereas the Logistic model is effective for symmetrical growth patterns. Polynomial models, although adaptable, had inadequate biological interpretability and faced challenges in accurately predicting the capacity for carrying. Different cultivation conditions markedly affected model efficacy, with mechanistic models exhibiting enhanced resilience under harsh settings. These findings emphasize the significance of using biologically relevant models for predicting microbial growth and offer critical insights for applications in microbial ecology, predictive microbiology, and bioprocess optimization \citep{Huber2017}.

\newpage
\section{Introduction}
Investigating population growth rates is crucial in biological research for comprehending ecosystem dynamics and forecasting the effects of environmental changes. In particular, pathogen transmission, antibiotic development, and the ecological roles of environmental microorganisms are influenced by microbial growth rates \citep{Peleg2011}. These single-celled organisms serve as exceptional models for analyzing biological development dynamics due to their rapid reproduction and responsiveness to environmental factors \citep{Pla2015}.

The expansion of microbial populations typically follows a sigmoidal pattern, identified by several distinct phases: lag phase (where cells adapt to their surroundings), exponential growth phase (characterized by rapid population increase), stationary phase (where growth stabilizes due to resource constraints), and occasionally a decline phase \citep{Lobete2015}. Accurate prediction of bacterial growth is necessary in various scientific and industrial fields, including food safety and disease management.

Previous research has proposed multiple mathematical methodologies to model these growth patterns. Linear models, such as quadratic and cubic polynomials, provide simplicity and computational efficiency but may lack biological significance. Nonlinear models, such as the Logistic and Gompertz models, incorporate biologically relevant parameters (e.g., maximum growth rate and carrying capacity) that better reflect the biological mechanisms driving population growth \citep{Huang2013}.

I systematically evaluates and compares four mathematical models—two linear (Quadratic and Cubic) and two nonlinear (Logistic and Gompertz)—for forecasting bacterial growth patterns. Utilizing data from global laboratory trials (LogisticGrowthData.csv), which comprises microbial biomass or cell count measurements over time for various bacterial species under diverse conditions, I aim to identify the models that most accurately represent the complex dynamics of microbial population increase.

Applying statistical criteria such as the Akaike Information Criterion (AIC), Bayesian Information Criterion (BIC), and R-squared, I focuses on the following research questions:
\begin{enumerate}
    \item How do nonlinear models compare with linear models in their ability to fit bacterial growth data across diverse species and environmental conditions?
    \item Does the Gompertz model, with its additional parameter for lag time, provide a significantly better fit than the Logistic model for bacterial growth curves that exhibit pronounced lag phases?
    \item To what extent do the conclusions about model superiority differ when using different statistical metrics (AIC versus BIC)?
\end{enumerate}

The findings from this investigation will enhance our understanding of which mathematical frameworks best represent bacterial growth dynamics and provide evidence-based guidance for model selection in practical applications spanning food safety, pharmaceutical development, and microbial ecology.

\section{Methods}

\subsection{Data Cleaning and Visualization}
\subsubsection{Data Cleaning Process}
The data preparation process began with the loading of the primary dataset (LogisticGrowthData.csv) and the metadata (LogisticGrowthMetaData.csv) utilizing RStudio. Following the assessment of fundamental information such as column names and unit consistency, the data cleaning phase concentrated on transforming both the Time and PopBio variables into numeric format while removing invalid entries. For Time, this involved excluding NA values and retaining only positive values; likewise, for PopBio, only positive values were preserved to maintain biological significance. Feature engineering required the creation of a log-transformed population biomass column (logPopBio) to enhance the visualization of growth patterns. Each experimental condition was designated a distinct ID by integrating Species, Temperature, Medium, and Citation information. The dataset was subsequently divided into 285 distinct subsets according to these IDs and preserved as individual CSV files within a structured directory framework.

\subsubsection{Data Visualization}
The visualization process utilized ggplot2 to generate six distinct graphical representations of bacterial growth dynamics. These included population growth curves plotting PopBio against Time, log-transformed growth analysis plotting logPopBio versus Time, temperature effects on biomass, comparisons across different growth media, temperature-time interaction effects, and species comparisons. Each visualization displayed unique patterns in the response of population biomass to diverse environmental factors and highlighted significant variations in their growth characteristics across temperature ranges and growth media.

\subsection{Mathematical Models}
\subsubsection{Quadratic Growth Model (Poly-2)}
A polynomial model allowing for moderate non-linearity:
\begin{equation}
N(t) = \beta_0 + \beta_1 t + \beta_2 t^2
\end{equation}
Where:
\begin{itemize}
    \item $N(t)$ is the population or biomass (PopBio) at time $t$.
    \item $\beta_0, \beta_1, \beta_2$ are polynomial coefficients.
\end{itemize}

\subsubsection{Cubic Growth Model (Poly-3)}
A higher-order polynomial model allowing for complex growth patterns:
\begin{equation}
N(t) = \beta_0 + \beta_1 t + \beta_2 t^2 + \beta_3 t^3
\end{equation}
Where:
\begin{itemize}
    \item $N(t)$ is the population or biomass (PopBio) at time $t$.
    \item $\beta_0, \beta_1, \beta_2, \beta_3$ are polynomial coefficients.
\end{itemize}

\subsubsection{Logistic Growth Model}
A classic sigmoidal growth model:
\begin{equation}
N(t) = \frac{K}{1 + \left(\frac{K - N_0}{N_0}\right) e^{-r (t - t_{min})}}
\end{equation}
Where:
\begin{itemize}
    \item $N(t)$ is the population or biomass (PopBio) at time $t$.
    \item $N_0$ is the initial population or biomass.
    \item $K$ is the carrying capacity, representing the maximum sustainable population.
    \item $r$ is the intrinsic growth rate.
    \item $t_{min}$ is the minimum time in the dataset (used for standardization).
\end{itemize}

\subsubsection{Gompertz Growth Model}
A flexible sigmoidal growth model with an asymmetric shape:
\begin{equation}
N(t) = N_0 + (K - N_0) \cdot \exp\left(-\exp\left(\frac{r_{max}}{K - N_0} (t_{lag} - t) + 1\right)\right)
\end{equation}
Where:
\begin{itemize}
    \item $N(t)$ is the population or biomass (PopBio) at time $t$.
    \item $K$ is the carrying capacity.
    \item $r_{max}$ is the maximum growth rate.
    \item $N_0$ is the initial population or biomass.
    \item $t_{lag}$ represents the lag phase duration.
\end{itemize}

\subsection{Set Starting Value}
The implementation of a dynamic parameter initialization strategy was crucial for successful model fitting across diverse bacterial growth datasets. The \texttt{calculate\_init\_params} function was developed to analyze temporal patterns in each growth curve and derive biologically relevant starting values for model parameters.

This function operates through several key steps:
\begin{enumerate}
    \item Calculation of instantaneous growth rates by determining slopes between consecutive data points using differential analysis:
    \begin{equation}
    \text{growth rate} = \frac{\text{diff}(\text{PopBio})}{\text{diff}(\text{time})}
    \end{equation}
    \item Identification of the inflection point where growth rate reaches its maximum:
    \begin{equation}
    \text{max\_slope\_index} = \text{which.max} (\text{slopes})
    \end{equation}
    \item Extraction of fundamental growth parameters:
    \begin{itemize}
        \item $N_0$: Minimum observed population size, representing initial biomass.
        \item $N_{max}$: Maximum observed population size, approximating carrying capacity.
        \item $r_{max}$: Maximum instantaneous growth rate from the steepest portion of the curve.
    \end{itemize}
    \item Derivation of lag phase duration using the formula:
    \begin{equation}
    t_{lag} = \text{time}[\text{max\_slope\_index}] - \frac{\log(N_{max} / N_0)}{r_{max}}
    \end{equation}
\end{enumerate}
This equation calculates the period before the beginning of exponential growth, based on established principles of population dynamics. The dynamically established initial values significantly enhanced the convergence rates of nonlinear least squares (\texttt{nls}) in comparison to static initialization methods. The algorithm prevented biologically implausible parameter estimates and minimized convergence failures by adjusting initial parameter estimates to the specific features of each dataset \citep{LeNovere2006}.

\subsection{Model Fitting}
The model fitting process was executed by the \texttt{fit\_models} function, which methodically applies four distinct growth models (Logistic, Gompertz, Quadratic, and Cubic) to each dataset. Every model fitting attempt is wrapped within a \texttt{tryCatch} block to manage possible convergence failures. 

Error handling was executed with \texttt{tryCatch} to address instances where model fitting encountered failures due to convergence problems or data constraints. The function utilizes nonlinear least squares (\texttt{nls}) with dynamically determined initial parameters and extended iteration limits ($\text{maxiter} = 1000$) to optimize fitting efficiency. This comprehensive method enabled us to systematically analyze all 285 unique bacterial growth datasets, assessing model performance by statistical metrics and visual evaluation of fit quality.

\subsection{Model Evaluation and Indicators}
To systematically evaluate and compare the performance of the four growth models across all datasets, three key statistical indicators were employed:

\subsubsection{Akaike Information Criterion (AIC)}
The AIC balances model fit quality against complexity using the formula:
\begin{equation}
AIC = 2k - 2 \ln(L)
\end{equation}
where $k$ represents the number of parameters and $L$ is the maximum likelihood value. Lower AIC values indicate preferred models, as they achieve better fit with fewer parameters.

\subsubsection{Bayesian Information Criterion (BIC)}
The BIC imposes a stricter penalty on model complexity than AIC:
\begin{equation}
BIC = k \ln(n) - 2 \ln(L)
\end{equation}
where $n$ is the number of observations. This criterion is more conservative in model selection, particularly favoring simpler models with larger sample sizes.

\subsubsection{Coefficient of Determination ($R^2$)}
The $R^2$ metric quantifies the proportion of variance explained by the model:
\begin{equation}
R^2 = 1 - \frac{SS_{res}}{SS_{tot}}
\end{equation}
where $SS_{res}$ is the sum of squared residuals and $SS_{tot}$ is the total sum of squares. $R^2$ values range from 0 to 1, with higher values indicating better explanatory power.

These indicators were methodically computed for all successful model fits, enabling a thorough comparison of model performance across the bacterial growth datasets. To determine which models consistently outperformed each criterion, statistical summaries were produced. This information helped determine the best mathematical formulations for characterizing the dynamics of bacterial growth under various circumstances \citep{Li2010}.

\subsection{Computing Languages and Tools}
For the project, I have chosen R as the primary programming language due to its robust capabilities for statistical analysis and specialized functions for nonlinear modeling. R's extensive ecosystem of packages for data science makes it particularly suitable for analyzing biological growth data.

I utilized a suite of powerful libraries in this project:
\begin{itemize}
    \item \textbf{Tidyverse}: This collection of packages (including \texttt{dplyr} and \texttt{tidyr}) provides fundamental tools for data manipulation. These tools were used to clean the bacterial growth dataset by eliminating invalid values, generating unique identifiers for growth curves based on species, temperature, medium, and citation details, and applying log transformations to population biomass values \citep{Pankowski2016}.
    \item \textbf{ggplot2}: An advanced data visualization package utilized to generate detailed visual representations of bacterial growth curves and model fittings. This enabled us to visually evaluate model efficiency across various bacterial species and conditions, generating plots for population growth over time, log-transformed growth trends, and the influence of temperature and medium on growth \citep{Tahon2018}.
    \item \textbf{nlstools}: This package enhances the capabilities of nonlinear least squares fitting by providing supplementary tools for model diagnostics. It was employed to assess the quality of model fits across various bacterial species and growth conditions. Custom control parameters (\texttt{nls.control}) were modified to enhance convergence during model fitting \citep{Wadsworth2017}.
\end{itemize}

The workflow was engineered for reproducibility, incorporating automated batch processing of datasets and systematic result storage for comprehensive analysis and comparison of model performance.

\section{Results}

\subsection{Model Fitting Success Rates}
The four growth models demonstrated varying degrees of fitting success across the 285 bacterial growth datasets. The Quadratic model showed the highest success rate (283 datasets, 99.3\%), followed by the Cubic model (269 datasets, 94.4\%). The Logistic model successfully fit 230 datasets (80.7\%), while the Gompertz model had the lowest success rate, fitting 122 datasets (42.8\%). These results suggest that polynomial models generally exhibited greater flexibility in capturing diverse growth patterns compared to the mechanistic models.

\begin{table}[h]
    \centering
    \caption{Model Fitting Success Counts}
    \begin{tabular}{lcc}
        \toprule
        Model & Success Count & Success Rate \\
        \midrule
        Cubic & 269 & 94.4\% \\
        Gompertz & 122 & 42.8\% \\
        Logistic & 230 & 80.7\% \\
        Quadratic & 283 & 99.3\% \\
        \bottomrule
    \end{tabular}
\end{table}

\subsection{Visual Inspection of Model Fits}
Visual examination of the fitted curves revealed distinct patterns in how each model captured growth dynamics. Figures 1-3 illustrate representative cases where models fit the data with varying accuracy. These visualizations show how different models capture the key growth phases: lag, exponential growth, and stationary phases.
\begin{figure}[H]
    \centering
    \includegraphics[width=0.8\textwidth]{../result/results/subset14.csv_plot}
    \caption{Example of model fitting comparison for a bacterial growth dataset (Fit two models)}
    \label{fig:fit_two_models}
\end{figure}
\begin{figure}[H]
    \centering
    \includegraphics[width=0.8\textwidth]{../result/results/subset8.csv_plot}
    \caption{Example of model fitting comparison for a bacterial growth dataset (Fit three models)}
    \label{fig:fit_three_models}
\end{figure}
\begin{figure}[H]
    \centering
    \includegraphics[width=0.8\textwidth]{../result/results/subset16.csv_plot}
    \caption{Example of model fitting comparison for a bacterial growth dataset (Fit four models)}
    \label{fig:fit_four_models}
\end{figure}
The Gompertz model typically provided superior representation of sigmoid growth patterns with distinct lag, exponential, and stationary phases, particularly in datasets exhibiting pronounced lag phases. The Logistic model performed well for symmetrical growth curves with less pronounced lag phases, while polynomial models (Quadratic and Cubic) showed better performance in fitting more complex or asymmetrical growth patterns, though they often failed to capture the biological constraints of population growth.

\subsection{Comprehensive Model Comparison}
Among the 105 datasets (36.8\%) where all four models successfully converged, a systematic comparison using information criteria revealed clear performance differences. Based on AIC, the Gompertz model was identified as the optimal choice for 58 datasets (55.2\%), followed by the Logistic model (26 datasets, 24.8\%), the Cubic model (17 datasets, 16.2\%), and the Quadratic model (4 datasets, 3.8\%). BIC analysis yielded similar results, with the Gompertz model preferred in 57 cases (54.3\%), Logistic in 27 cases (25.7\%), Cubic in 17 cases (16.2\%), and Quadratic in 4 cases (3.8\%).

\begin{table}[h]
    \centering
    \caption{Best Model Counts Based on Information Criteria}
    \begin{tabular}{lcc}
        \toprule
        Model & Best AIC Count & Best BIC Count \\
        \midrule
        Cubic & 17 & 17 \\
        Gompertz & 58 & 57 \\
        Logistic & 26 & 27 \\
        Quadratic & 4 & 4 \\
        \bottomrule
    \end{tabular}
\end{table}

\subsection{Statistical Performance Metrics}
Examination of mean information criteria values further supported these findings. The Gompertz model achieved the lowest mean AIC (2.20), indicating superior performance, followed closely by the Logistic model (2.95). Both mechanistic models substantially outperformed the polynomial models, with the Quadratic model showing the highest mean AIC (19.75) and the Cubic model intermediate (11.92). Mean BIC values followed a similar pattern, with Logistic (4.64) and Gompertz (4.80) models performing considerably better than the Cubic (14.56) and Quadratic (21.69) alternatives.

The coefficient of determination ($R^2$) analysis provided additional validation, with the Gompertz model achieving the highest mean $R^2$ value (0.957), followed closely by the Logistic model (0.956). The Cubic and Quadratic models demonstrated lower explanatory power with mean $R^2$ values of 0.907 and 0.852, respectively.

\begin{table}[h]
    \centering
    \caption{Model Performance Metrics}
    \begin{tabular}{lccc}
        \toprule
        Model & Mean AIC & Mean BIC & Mean $R^2$ \\
        \midrule
        Cubic & 11.92123 & 14.55602 & 0.9068 \\
        Gompertz & 2.195273 & 4.795892 & 0.9574 \\
        Logistic & 2.946791 & 4.642938 & 0.9557 \\
        Quadratic & 19.7539 & 21.69178 & 0.8518 \\
        \bottomrule
    \end{tabular}
\end{table}

\section{Discussion}

\subsection{Addressing Research Questions}
The analysis of 285 bacterial growth datasets provides clear answers to my initial research questions. Nonlinear mechanistic models outperformed linear polynomial models in statistical fit despite lower fitting success rates. The Gompertz model, with its additional parameter for lag time, provided better fits than the Logistic model for most bacterial growth curves, particularly those with pronounced lag phases. Remarkably, these conclusions remained consistent across different statistical metrics (AIC and BIC), reinforcing the robustness of our findings.

\subsection{Biological Relevance and Model Extensions}
The superior performance of mechanistic models stems from their biological relevance, as they incorporate parameters representing actual growth processes, whereas polynomial models merely approximate curve shapes without biological meaning. The Gompertz model particularly excels by capturing the asymmetric nature of bacterial growth and explicitly accounting for the lag phase.

A significant constraint of traditional growth models is their failure to accurately represent the decline phase that occurs when bacterial populations encounter limited resources or the accumulation of toxic metabolites \citep{Biller2015}. Enhanced models might reduce this limitation by integrating parameters for population decline after peak growth. These extensions must account for the timing of peak population ($t_{peak}$), the rate of decline, and the relationship between growth rate, carrying capacity, and decline dynamics. These enhancements would produce more accurate representations of entire bacterial lifecycles, which is crucial for long-term research or resource-constrained settings \citep{Cusick2020}. Moreover, considerations of numerical stability are important when applying these extended models to avoid computational problems associated with exponential terms and to guarantee dependable parameter estimation across various growth patterns \citep{Fadeev2016}.

\subsection{Study Limitations}
Several limitations of our research should be acknowledged:

\begin{itemize}
    \item \textbf{Data Source Limitations:} Laboratory growth data may not fully reflect microbial growth in natural environments, where additional factors influence population dynamics.
    \item \textbf{Environmental Factors:} Although temperature was documented, we did not thoroughly investigate its impact on model performance across various bacterial species. Additional environmental variables (pH, oxygen concentrations, nutrient availability) may affect growth patterns in manners not represented by our models \citep{Huber2017}.
    \item \textbf{Computational Challenges:} The Gompertz model, while statistically superior, had the lowest fitting success rate (42.8\%), suggesting parameter estimation difficulties for this more complex model. It may be due to its higher sensitivity to initial parameter estimation and the presence of local optima in nonlinear least squares fitting. Future research could explore alternative optimization techniques, such as Bayesian inference or Markov Chain Monte Carlo (MCMC), to improve convergence.
    \item \textbf{Model Simplification:} The models analyzed are simplifications of the complicated biological reality of bacterial growth, which may incorporate more complex dynamics such as diauxic growth, quorum sensing, and adaptive responses \citep{Ivanova2015}.
\end{itemize}

These limitations indicate prospects for future research, such as the creation of more robust initialization techniques for nonlinear model fitting and the integration of supplementary environmental variables into growth forecasts.


\bibliographystyle{apalike}
\bibliography{mybibs}

\end{document}
